% !TeX encoding = UTF-8
% !TEX root = ./presentation.tex
% !TEX spellcheck = pt_BR

   \subsection{Ferramenta {\it Profile}}
   %\frame{\centering \bf \Huge \color{beamerCinza} \textit{Profile}}

      \begin{frame}{\textit{Profile}} \vspace{-1em}
         
         \begin{columns}
            \begin{column}{0.5\textwidth}
               
               \begin{figure}[h] \centering
                  \vspace{-24pt}
                  \includegraphics[width=0.8\textwidth]{img/f4-1-2.png}
                  \caption{\Profile\ da codificação de imagem em formato JPEG. Fonte: \cite{Sass2010}.}
               \end{figure}
            \end{column}
            \begin{column}{0.5\textwidth}
               \vspace{-1cm}
               \begin{itemize}
                  \item Procedimento da ferramenta
                  \begin{enumerate}
                     \setlength{\itemsep}{0.8em}
                     \item \textbf{Realiza-se interrupções periódica} no programa; e
                     \item Amostra o seu \textit{program counter}.
                     
                     \item Utiliza-se de um histograma para contar o endereço particular;
                     
                     \item \textbf{Calcula a fração aproximada do tempo} total de execução \textbf{gasto em suas partes}. 
                  \end{enumerate}
               \end{itemize}
            
            \end{column}
         \end{columns}
         \pdfnote{OLHAR OS PARENTESES (\%)}
         \pdfnote{coletar info em tempo de execução.}
         \pdfnote{soft entrada, mensura-se seu tempo.}
         
      \end{frame}


   \subsection{Sistemas Computacionais Wearables}
      %\frame{\centering \bf \Huge \color{beamerCinza} \textit{Sistemas Computacionais \Wearables}}

      \begin{frame}{\Wearables} \vspace{-1em}
         \begin{itemize} \setlength{\itemsep}{1.4em}
            \item Definição
            \begin{itemize} \setlength{\itemsep}{0.4em}
               \item Feito de subconjunto de componentes;
               \item Possibilidade de ter recursos sensoriais e escaneamentos;
               \item Requer serviço autônomo, contínuo, em um longo período de tempo.
               \item Integrar-se ao sistema corporal;
               \item Podendo expandir suas capacidades;
               \item Ou seja, são \textbf{embutidos inseridos em ambiente \mobile\ de seus usuários, não exercendo a mesma atividade} \cite{Plessl2003}. 
            \end{itemize}
         
            %\item Acesso constante, conveniente, portátil e principalmente \textit{hands-free}.
            
            \item Em 2015, foi previsto um total de 6,5 bi de dispositivos ativamente conectados \cite{RobvanderMeulen2015}
            \begin{itemize} \setlength{\itemsep}{0.4em}
               \item Cerca de 20\% da população possui pelo menos um dispositivo sendo que 10\% utiliza-o todos os dias \cite{lee2016information};
               \item \textbf{Tendência:} Superar dispositivos manuais.
            \end{itemize}
            
         \end{itemize}
         \pdfnote{+ integração entre tecnologia e s.humano}
      \end{frame}  
     
      
      \begin{frame} {\Wearables}{Algumas Definições Científicas} \vspace{-1em}
         % Introdução histórica e geral
         \begin{block}{Segundo \cite{Amorim2017}:} 
            Com a possibilidade de ter um computador acoplado ao corpo, \textbf{proporciona ao usuário um nível de informações} contextualizadas \textbf{dentro de um ambiente interativo}.
         \end{block}
            
            \bigskip
      
         \begin{block}{Segundo \cite{Gemperle1998}:} 
            Dispositivo que possui sua `\textit{wearability}'. Este definido como a \textbf{interação entre o corpo humano e o objeto \textit{wearable} estendendo ao corpo em movimento}.
         \end{block}
      
      \end{frame}
   
      \begin{frame}{\Wearables}{Situação Exemplo} \vspace{-1em}
         \begin{columns}
            \begin{column}{0.6\textwidth}
               \begin{figure}[h] \centering
                  \includegraphics[width=1\textwidth]{img/into-wearable2.png}
                  \caption{Exemplificação de alguns dispositivos \wearables. Fonte: \cite{Plessl2003}.}
                  \label{fig:into-wearable}
               \end{figure}
            \end{column}
            \begin{column}{0.4\textwidth}
               \begin{itemize}
                   \setlength{\itemsep}{1.0em}
                  \item Distribuição espacial dos módulos pelo corpo;
                  
                  \item Comunicação:
                  \begin{itemize}
                     \setlength{\itemsep}{0.5em}
                     \item Deve ser avaliada energeticamente \cite{Kymissis1998}.
                     \item \textbf{Pode ser mista, sendo a predominância sem-fio pela mobilidade} \cite{Plessl2003}.
                  \end{itemize}
               \end{itemize}
               
            \end{column}
         \end{columns}
         
      \end{frame}
      
      \begin{comment}
         % Característica de um dispositivo wearable
         \begin{frame}{Características de um \Wearable} \vspace{-1em}
            \begin{itemize} \setlength{\itemsep}{1.5em}
               \item Caracteriza-se um \wearable\ acordando \textbf{às suas funcionalidades e requisitos de \hardware}\ \cite{Delabrida2016, Amorim2017}.
               
               \item Sendo essas:
                
               \begin{itemize}
                  \setlength{\itemsep}{0.5em}
                  \item Soluções em \hardware\ \textbf{compartilham uma arq. e org. interna} de recursos comum.
                  
                  \item Também podem ser expandidos às \textbf{características de OS}.
                  
               \end{itemize}
            \end{itemize}
      
            \begin{figure}[h] \centering
               \vspace{-5pt}
               \includegraphics[width=0.9\textwidth]{img/rt-gradiente.png}
               %\vspace{-10pt}
               \caption{Classificação de \wearables. Fonte: Adaptado de \cite{Amorim2017}.}
            \end{figure}
         \end{frame}
      
      \begin{frame}{\Wearable}{Sistemas Operacionais} \vspace{-1em}
            
         \begin{itemize} \setlength{\itemsep}{1.5em}
            \item São comumente \textbf{focado em um único tipo de seguimento} de produto
            \begin{itemize}
               \item Como os \textit{smartwatches}.
            \end{itemize}
            
            \item Vantagens
            \begin{itemize}
               \item Proporcionam aos desenvolvedores um
               \begin{itemize}
                  \item Meio para sua aplicação final; além de
                  \item Produto de alta qualidade.
               \end{itemize}
            \end{itemize}
            
            \item Desvantagens
            \begin{itemize}
               \item Atualmente não existe \textbf{nenhum sistema que satisfaça todos os requisitos} de todas as classificações \cite{Amorim2017}.
            \end{itemize}
         \end{itemize}
      \end{frame}
      
      
      \begin{frame}{\Wearables}{Características a serem Consideradas no \Design} \vspace{-1em}
         
         \begin{itemize}
            \setlength{\itemsep}{0.9em}
            \item \textbf{Performance de multi-nós:} 
            \begin{itemize} \setlength{\itemsep}{0.4em}
               \item \textbf{Tarefas baixa demanda computacional ou nem restrições de tempo rigorosa}: Requer uma performance base fixa;
               
               \item Consideração de restrições de \textbf{tempo-real}.
            \end{itemize}
            
            \item \textbf{Gasto energético consciente:} 
            \begin{itemize} \setlength{\itemsep}{0.4em}
               \item Manter-se ativo e funcional num certo período de tempo;
               \item Gerenciamento do gasto de energia.
               
            \end{itemize}
            
            \item \textbf{Flexibilidade:}
            \begin{itemize} \setlength{\itemsep}{0.4em}
               \item Útil em situações altamente dinâmicas.
               \begin{itemize}
                  \item Pode variar de acordo com as escolhas do usuário ou também com o contexto e local utilizado;
                  \item Trocas de roupa.
               \end{itemize}
               \item Critérios: confiabilidade, disponibilidade e itens dependentes de sua forma como volume e peso.
            \end{itemize}
         \end{itemize}
      \end{frame}
      \end{comment}
      
      
      \begin{frame}{\Wearable\ + FPGA}{Justificativa  \cite{Plessl2003}} \vspace{-1em}
         \begin{itemize}
            \setlength{\itemsep}{1.6em}
            \item \Wearables\ necessitam
            \begin{itemize}
               \setlength{\itemsep}{1.0em}
               
               \item \textbf{Requisitos de alta performance e consumo de energia consciente:} demandam um sistema computacional econômico;
               
               \item \textbf{Requisitos flexíveis:} demanda um sistema de computação programável de propósito geral.
            \end{itemize}
         
            \item Ao utilizar de um \hardware\ reconfigurável nos permite alcançar
            \begin{itemize}
                \setlength{\itemsep}{1.0em}
               \item \textbf{Alto processamento};
               \item \textbf{Com maior eficiência energética} comparando com processadores para computação intensiva em tempo real;
               \item Isso, junto com a disponibilidade de circuito reconfigurável para síntese.
            \end{itemize}
         \end{itemize}
      \end{frame}
